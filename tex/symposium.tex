\documentclass{beamer}

\title{On the Existence of Double-Descent in Reinforcement Learning}
\author{Richard Harnisch}
\date{27 January 2026}

\begin{document}

\begin{frame}
    \titlepage
\end{frame}

\begin{frame}{Outline}
    \tableofcontents
\end{frame}


% - Show results, unfortunately it doesn't show up

\section{Introduction}
\begin{frame}{What is Double Descent?}
    % Describe phenomenon of DD in SL and the paradox
\end{frame}

\begin{frame}{DD in Reinforcement Learning}
    % has not been studied much in RL...
\end{frame}

\section{Methods}
\begin{frame}{Methods}
    % - Move to methods
    % - Explain that to mimick SL we need training and validation maps, need something to plot on the y axis of these plots because we dont have the loss
    % Explain FIM, RL setup, environments, agents, training/validation maps
\end{frame}

\section{Results \& Discussion}
\begin{frame}{Results}
    % Results, unfortunately it doesn't show up
\end{frame}

\begin{frame}{Discussion}
    % Discussion
\end{frame}

\begin{frame}{Limitations \& Future Work}
    % Limitations & Future Work
\end{frame}

\section{Conclusion}
\begin{frame}{Conclusion}
    % Conclusion
\end{frame}

\end{document}