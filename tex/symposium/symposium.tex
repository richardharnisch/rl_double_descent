\documentclass{beamer}

\title{On the Existence of Double-Descent in Reinforcement Learning}
\author{Richard Harnisch}
\date{27 January 2026}
\usepackage[backend=biber,style=authoryear]{biblatex}
\addbibresource{main.bib}

\begin{document}

\begin{frame}
    \titlepage
\end{frame}

\begin{frame}{Outline}
    \tableofcontents
\end{frame}

\section{Double Descent Background}
\begin{frame}{What is Double Descent?}
    \begin{itemize}
        \item \textbf{Double Descent (DD)} is a phenomenon of models recovering generalization capacity after surpassing the interpolation threshold \parencite{nakkiran2019deepdoubledescentbigger}.
        \item Classically, increasing model capacity leads to a U-shaped risk curve:
              \begin{figure}
                  \centering
                  \includegraphics[height=0.4\textheight]{images/u-shaped.pdf}
                  \caption{Classical U-shaped risk curve arising from the bias-variance trade-off. \textcite{Belkin_2019}.}
                  \label{fig:u-shaped}
              \end{figure}
    \end{itemize}
\end{frame}

\begin{frame}{Double Descent Risk Curve}
    \begin{itemize}
        \item However, modern machine learning models often operate in regimes where they can perfectly fit (interpolate) the training data.
        \item This can occur in both model size and training time regimens as well as multiple problem types in supervised learning \parencite{Belkin_2019, nakkiran2019deepdoubledescentbigger}.
    \end{itemize}
    \begin{figure}
        \centering
        \includegraphics[height=0.35\textheight]{images/doubledescent.pdf}
        \caption{Double Descent risk curve exhibiting a second descent. \textcite{Belkin_2019}.}
        \label{fig:double-descent}
    \end{figure}
\end{frame}

\begin{frame}{DD in Reinforcement Learning}
    \begin{itemize}
        \item Double Descent has not been studied much in Reinforcement Learning (RL) settings.
        \item There is some research on theoretical aspects of overparameterization in RL with regard to visited states \parencite{brellmann2024doubledescentreinforcementlearning}.
        \item There is also an approach of using information theoretic metrics to analyze generalization in RL \parencite{veselý2025presencedoubledescentdeepreinforcement}.
    \end{itemize}
    % has not been studied much in RL...
\end{frame}

\section{Experimental Setup}
\begin{frame}{Methods}
    \begin{itemize}
        \item We are trying to emulate the supervised learning setup in RL: Train/test split, and plotting a ``risk'' curve.
        \item To mimick train and test sets, we use a family of seeded randomly generated maps, training on a subset of these maps and testing on a disjoint subset.
    \end{itemize}

    \begin{figure}
        \centering
        \begin{minipage}{0.32\textwidth}
            \fbox{\includegraphics[width=\linewidth]{images/env1.png}}
        \end{minipage}
        \hfill
        \begin{minipage}{0.32\textwidth}
            \fbox{\includegraphics[width=\linewidth]{images/env2.png}}
        \end{minipage}
        \hfill
        \begin{minipage}{0.32\textwidth}
            \fbox{\includegraphics[width=\linewidth]{images/env3.png}}
        \end{minipage}
        \caption{Examples of randomly generated train and test environments, with agent (blue) and goal (green) tiles.}
        \label{fig:env-examples}
    \end{figure}
    % - Move to methods
    % - Explain that to mimick SL we need training and validation maps, need something to plot on the y axis of these plots because we dont have the loss
    % Explain FIM, RL setup, environments, agents, training/validation maps
\end{frame}

\begin{frame}{Metrics}
    \begin{itemize}
        \item To plot risk curves, we need something to plot on the y-axis since we don't have a loss function in RL.
        \item We use the \textbf{average return} over the test maps as our metric.
        \item Further, we compute the trace of the \textbf{Fisher Information Matrix (FIM)} of the policy network.
        \item The FIM is defined as:
              \begin{equation*}
                  F = \mathbb{E}_{a, s \sim \pi_\theta} \left[ \nabla_\theta \log \pi_\theta(a|s) \nabla_\theta \log \pi_\theta(a|s)^T \right]
              \end{equation*}
              where $s, a$ are sampled from trajectories under the policy $\pi_\theta$, collected on the training maps.
        \item Thus the FIM is the expected outer product of the score function (the gradient of the log-likelihood).
        \item The trace of the FIM gives a measure of the sensitivity of the policy to parameter changes, which can be related to generalization.
    \end{itemize}
\end{frame}

\begin{frame}{Agents \& Training Setup}
    \begin{itemize}
        \item We train two agent types: A DQN \parencite{mnih2013playingatarideepreinforcement} agent and a TRPO agent \parencite{schulman2017trustregionpolicyoptimization}, both using feedforward neural networks.
        \item DQNs implement Q-learning with experience replay using a neural network to estimate Q-values.
        \item TRPO is a policy gradient method that optimizes the policy directly while constraining updates using a trust region defined by the Kullback-Leibler (KL) divergence.
        \item We vary the model capacity by changing the depth and width of the networks.
        \item Training is performed on a fixed set of training maps, with evaluation held-out test maps.
        \item We record the average return and trace of the FIM at various points during training to analyze the presence of Double Descent.
    \end{itemize}
\end{frame}

\section{Results \& Discussion}
\begin{frame}{Results in Parametric Regime}
    \begin{figure}
        \centering
        \includegraphics[height=0.8\textheight]{images/results_capacity.png}
        \caption{Results from TRPO agents trained on 100 maps for 50k episodes.}
        \label{fig:results-example}
    \end{figure}
\end{frame}

\begin{frame}{Results in Episodic Regime}
    \begin{figure}
        \centering
        \includegraphics[height=0.8\textheight]{images/episodes_capacity.png}
        \caption{Results of a TRPO agent trained on 100 maps for 1.8M episodes.}
        \label{fig:episodes-example}
    \end{figure}
\end{frame}

\begin{frame}{Discussion}
    \begin{itemize}
        \item Unfortunately, we are unable to observe a clear Double Descent phenomenon in our experiments.
        \item There may be several reasons for this, including the complexity of reinforcement learning environments and the computational constraints limiting the scale of our experiments.
        \item We observe a spike in the FIM trace around the interpolation threshold. This indicates that the model is unstable in the parameter space, but stabilizes later.
        \item However, this does not translate to a clear second ascent in average return.
    \end{itemize}
    % Discussion
\end{frame}

\section{Limitations \& Future Work}
\begin{frame}{Limitations \& Future Work}
    \begin{itemize}
        \item We are constrained by limited access to compute, with training runs taking multiple days and wait times for available GPUs.
        \item One avenue for further work is exploring the transition between lacking generalization and perfect generalization---we observe no emerging generalization gap when training on 1,000 maps.
        \item Additionally, we could explore why the FIM trace's spike around the interpolation threshold comes back down despite the test return not moving.
        \item Further, training on larger maps as well as different algorithms could be explored, as well as stochastic transitions.
    \end{itemize}
\end{frame}

\section{Questions}
\begin{frame}[allowframebreaks]{References}
    \printbibliography
\end{frame}

\begin{frame}{Thank You!}
    \begin{center}
        Questions?
    \end{center}
\end{frame}

\begin{frame}{Environment Details}
    \begin{itemize}
        \item Each map is $8\times 8$ tiles, with the agent in the top left and the goal in the bottom right unless randomized.
        \item Every other tile has a 20\% chance of being a wall, with a possible path being guaranteed.
        \item At each timestep, the agent receives a reward given by:
              \begin{equation*}
                  r_t =
                  \begin{cases}
                      1,                                & \text{if goal reached at time } t \\
                      \Phi(s_{t+1}) - \Phi(s_t) - 0.01, & \text{otherwise}
                  \end{cases}
              \end{equation*}
              where $\Phi(s) = \frac{\text{Euclidean distance to goal}}{100}$ is the potential function.
        \item Episodes are truncated at 64 timesteps and terminated when the goal is reached.
        \item Action space: {up, right, down, left}, deterministic transitions.
        \item Observation space: $8\times 8$ grid with one-hot encoded tiles, frame stacking of 2 frames ($8\times 8\times 4\times 2 = 512$ bits).
    \end{itemize}
\end{frame}

\begin{frame}{Training Configurations}
    \begin{itemize}
        \item On the following configurations, we trained agents of varying depths between 3 and 8 and widths between 4 and 1024:
              \begin{tabular}{l|c|c|l}
                  Type & Train Maps & Training Episodes & Start \& Goal Position                                            \\
                  \hline
                  TRPO & 10         & 10,000            & Standard\footnote{Agent starts in top left, goal in bottom right} \\
                  DQN  & 50         & 20,000            & Standard                                                          \\
                  DQN  & 100        & 50,000            & Standard                                                          \\
                  TRPO & 50         & 20,000            & Randomized                                                        \\
                  TRPO & 100        & 50,000            & Randomized                                                        \\
                  TRPO & 500        & 1,000,000         & Standard                                                          \\
              \end{tabular}
        \item On the following configurations, we trained agents of depth 3 and widths 4, 16, 64, 256, and 1024 for unlimited time:
              \begin{tabular}{l|c|l}
                  Type & Train Maps & Start \& Goal Position \\
                  \hline
                  TRPO & 50         & Standard               \\
                  TRPO & 100        & Standard               \\
              \end{tabular}
    \end{itemize}
\end{frame}

\begin{frame}{The Fisher Information Matrix}
    \begin{itemize}
        \item The \textbf{Fisher Information Matrix} (FIM) (originally in \cite{fisher1921}, in this form in \cite{kakade2001}) is defined as:
              \begin{equation*}
                  F(\theta) = \mathbb{E}_{a, s \sim \pi_\theta} \left[ \nabla_\theta \log \pi_\theta(a|s) \nabla_\theta \log \pi_\theta(a|s)^T \right]
              \end{equation*}
        \item $s, a$ are sampled under the stochastic policy $\pi_\theta$ defined by parameters $\theta$.
        \item $\nabla_\theta \log \pi_\theta(a|s)$ is the gradient of the log-probability of the action given the state with respect to the policy parameters
        \item A FIM where $\theta = [\theta_1, \theta_2]^T$ is therefore a $2\times 2$ matrix given by
    \end{itemize}
    \begin{equation*}
        \mathbb{E}_{a, s \sim \pi_\theta}
        \begin{bmatrix}
            \left( \frac{\partial}{\partial \theta_1} \log \pi_\theta(a|s) \right)^2                                        & \frac{\partial}{\partial \theta_1} \log \pi_\theta(a|s) \frac{\partial}{\partial \theta_2} \log \pi_\theta(a|s) \\
            \frac{\partial}{\partial \theta_2} \log \pi_\theta(a|s) \frac{\partial}{\partial \theta_1} \log \pi_\theta(a|s) & \left( \frac{\partial}{\partial \theta_2} \log \pi_\theta(a|s) \right)^2
        \end{bmatrix}
    \end{equation*}
\end{frame}

\begin{frame}{Computing the FIM's Trace}
    \begin{itemize}
        \item We do not need to form the full $|\theta| \times |\theta|$ matrix!
        \item Recall $F = \mathbb{E}_{a, s \sim \pi_\theta} \left[ \nabla_\theta \log \pi_\theta(a|s) \nabla_\theta \log \pi_\theta(a|s)^T \right]$, and let
              $g = \nabla_\theta \log \pi_\theta(a|s)$. Then $F = \mathbb{E}[g g^T]$.
        \item Since the trace of a matrix is the sum of its diagonal entries $\sum_i F_{ii}$, we have (see \cite{Petersen2008}):
              \begin{equation*}
                  \operatorname{tr}(F) = \sum_i \mathbb{E}[g_i^2] = \mathbb{E}\left[\sum_i g_i^2\right] = \mathbb{E}[g^T g] = \mathbb{E}[\|g\|_2^2].
              \end{equation*}
        \item We can thus form a Monte Carlo estimate of the trace of the FIM by sampling states from collected trajectories and actions from the policy in that state:
              \begin{equation*}
                  \operatorname{tr}(F) \approx \frac{1}{N} \sum_{i=1}^N \| \nabla_\theta \log \pi_\theta(a_i | s_i) \|_2^2
              \end{equation*}
    \end{itemize}
\end{frame}

\begin{frame}{Algorithms: DQN and TRPO}
    \begin{itemize}
        \item \textbf{Deep Q-Networks (DQN)} \parencite{mnih2013playingatarideepreinforcement} are value-based and approximate the optimal action-value function $Q^*(s, a)$ using a neural network. The agent selects actions based on an $\epsilon$-greedy policy derived from the Q-values.
        \item \textbf{Trust Region Policy Optimization (TRPO)} \parencite{schulman2017trustregionpolicyoptimization} is a policy gradient method that optimizes the policy directly.
        \item TRPO uses a surrogate objective function given by the expected advantage $\mathbb{E}_{s, a \sim \pi_{\theta_\text{old}}} \left[ \frac{\pi_\theta(a|s)}{\pi_{\theta_\text{old}}(a|s)} A^{\pi_{\theta_\text{old}}}(s, a) \right]$.
        \item TRPO constrains the policy updates using a trust region defined by the Kullback-Leibler (KL) divergence to ensure stable learning by $D_{KL}(\pi_{\theta_\text{old}} || \pi_\theta) \leq \delta$ for some small $\delta$.
    \end{itemize}
\end{frame}

\end{document}